
This paper presents a study on the performance of XGBoost and SVM models in identifying negation cues in text. Negation is a complex semantic phenomenon that can significantly impact the interpretation of a sentence's meaning. The study utilizes the SEM 2012 shared task dataset, which contains annotated examples of negation cues. A comprehensive set of lexical and syntactic features is used in combination with the models to predict the presence of negation cues in sentences. The results show that XGBoost performs slightly better than SVM in terms of accuracy and F1 score. A qualitative error analysis is conducted to gain insights into the strengths and weaknesses of both models. The analysis reveals that both models struggle with multi-word negation cues, such as "not only" and "rather than." Overall, the study highlights the importance of considering the complexity of negation and using appropriate features and models to accurately identify negation cues in text.


\keywords{Support Vector Machine (SVM)  \and XGBoost \and Negation Cue Detection.}