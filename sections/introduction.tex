
Negation Cue Detection is an essential aspect of Natural Language Processing (NLP) and Text Mining that enables automated systems to classify the sentiment or opinion expressed in a sentence. In essence, negation cues are linguistic markers that indicate the presence of negation in a sentence. They can manifest in diverse forms, such as single words like "never," multi-words like "no longer" and "by no means," or affixes like "im-" and "-less." Negation cues can also be complex, discontinuous, and have a scope that encompasses all negated concepts and events, excluding the negation cue itself \cite{jbara-2012}.
\\

Accurately identifying negation cues is crucial for many NLP and Text Mining tasks, such as Sentiment Analysis \cite{cruz2016machine} and Clinical Text Mining \cite{mehrabi2015deepen}. because they can significantly impact the meaning of a sentence and affect the overall interpretation of the text. Therefore, detecting these cues is vital for extracting the correct sentiment or opinion expressed in the text data.
\\

This paper presents two models for automatic negation cue classification, namely Support Vector Machine (SVM) and eXtreme Gradient Boosting (XGBoost). An ablation study is conducted to determine the impact of the features used in automatic detection, and the best-performing models are selected for evaluation on unseen test data. The study aims to assess the performance of the models in accurately detecting negation cues and their scope, which is crucial for obtaining reliable insights from the text data.


