
Detecting negative sentences and information is an important task in the field of Natural Language Processing (NLP) and Text Mining. The detection of negation cues and their scope is crucial for a consistent interpretation of the meaning of natural language texts, and automatic classification approaches play a key role in supporting NLP tasks such as Sentiment Analysis \cite{cruz2016machine} and Clinical Text Mining \cite{mehrabi2015deepen}.

Negation cues can be single words (e.g., "never"), multiwords (e.g., "no longer," "by no means"), or affixes (e.g. "im-," "-less"). Negation cues can also be discontinuous, and their scope can include all negated concepts and events, while the negation cue itself is not part of the scope.


In this paper, we present an analysis of the performance of SVM and XGBoost classifiers for detecting negation. The classifiers are trained on different feature sets and their performance is evaluated through experiments. The paper is structured as follows: first, related work in negation cue detection is reviewed, then an annotation experiment that was carried out to identify sources of bias in data annotation is described. Next, the Negation Cue Detection experiment itself is presented, including the description of the dataset used for training and testing, feature extraction and selection, and the details of the SVM and XGBoost models. Finally, the main challenges and limitations of the given research and suggested directions for future work are discussed.

