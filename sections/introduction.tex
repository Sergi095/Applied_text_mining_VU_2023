Negation cue detection is a fundamental task in natural language processing, which involves identifying words or phrases that indicate the presence of negation in text. Negation cues are linguistic expressions that convey negation, such as "not," "never," "no one," and "nothing." Detecting negation cues is a crucial step in understanding the meaning of a sentence, as the presence of negation can drastically alter its interpretation. Negation cue detection is an important task in many natural language processing applications, including sentiment analysis, opinion mining, and question answering systems.

The goal of negation cue detection is to identify all the words and phrases in a sentence that signal the presence of negation. Negation cues can appear in various forms, such as adverbs, adjectives, nouns, and verbs, and can occur in different positions within a sentence. Some negation cues may also be expressed implicitly, through negation of other words or phrases. Negation cue detection typically involves the use of machine learning techniques, such as support vector machines (SVMs), decision trees, or conditional random fields (CRFs), which can automatically learn to recognize patterns and features in text that are indicative of negation.

Negation cue detection is a challenging task, as the presence of negation can depend on various contextual factors, such as the tense, mood, and aspect of the sentence, as well as the semantic relationships between words. Negation cues can also be ambiguous, and their interpretation may vary depending on the context. Therefore, effective negation cue detection requires the use of a combination of lexical, syntactic, and semantic features, as well as linguistic knowledge and common sense reasoning. The development of accurate and reliable negation cue detection systems is essential for improving the performance of natural language processing applications that involve text analysis.




% Detecting negative sentences and information is an important task in the field of Natural Language Processing (NLP) and Text Mining. The detection of negation cues and their scope is crucial for a consistent interpretation of the meaning of natural language texts, and automatic classification approaches play a key role in supporting NLP tasks such as Sentiment Analysis \cite{cruz2016machine} and Clinical Text Mining \cite{mehrabi2015deepen}.

% Negation cues can be single words (e.g., "never"), multiwords (e.g., "no longer," "by no means"), or affixes (e.g. "im-," "-less"). Negation cues can also be discontinuous, and their scope can include all negated concepts and events, while the negation cue itself is not part of the scope.


% In this paper, we present an analysis of the performance of SVM and XGBoost classifiers for detecting negation. The classifiers are trained on different feature sets and their performance is evaluated through experiments. The paper is structured as follows: first, related work in negation cue detection is reviewed, then an annotation experiment that was carried out to identify sources of bias in data annotation is described. Next, the Negation Cue Detection experiment itself is presented, including the description of the dataset used for training and testing, feature extraction and selection, and the details of the SVM and XGBoost models. Finally, the main challenges and limitations of the given research and suggested directions for future work are discussed.

